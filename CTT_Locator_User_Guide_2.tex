% Options for packages loaded elsewhere
\PassOptionsToPackage{unicode}{hyperref}
\PassOptionsToPackage{hyphens}{url}
%
\documentclass[
]{article}
\usepackage{lmodern}
\usepackage{amssymb,amsmath}
\usepackage{ifxetex,ifluatex}
\ifnum 0\ifxetex 1\fi\ifluatex 1\fi=0 % if pdftex
  \usepackage[T1]{fontenc}
  \usepackage[utf8]{inputenc}
  \usepackage{textcomp} % provide euro and other symbols
\else % if luatex or xetex
  \usepackage{unicode-math}
  \defaultfontfeatures{Scale=MatchLowercase}
  \defaultfontfeatures[\rmfamily]{Ligatures=TeX,Scale=1}
\fi
% Use upquote if available, for straight quotes in verbatim environments
\IfFileExists{upquote.sty}{\usepackage{upquote}}{}
\IfFileExists{microtype.sty}{% use microtype if available
  \usepackage[]{microtype}
  \UseMicrotypeSet[protrusion]{basicmath} % disable protrusion for tt fonts
}{}
\makeatletter
\@ifundefined{KOMAClassName}{% if non-KOMA class
  \IfFileExists{parskip.sty}{%
    \usepackage{parskip}
  }{% else
    \setlength{\parindent}{0pt}
    \setlength{\parskip}{6pt plus 2pt minus 1pt}}
}{% if KOMA class
  \KOMAoptions{parskip=half}}
\makeatother
\usepackage{xcolor}
\IfFileExists{xurl.sty}{\usepackage{xurl}}{} % add URL line breaks if available
\IfFileExists{bookmark.sty}{\usepackage{bookmark}}{\usepackage{hyperref}}
\hypersetup{
  pdftitle={CTT Locator User Guide - New},
  pdfauthor={David La Puma},
  hidelinks,
  pdfcreator={LaTeX via pandoc}}
\urlstyle{same} % disable monospaced font for URLs
\usepackage[margin=1in]{geometry}
\usepackage{graphicx,grffile}
\makeatletter
\def\maxwidth{\ifdim\Gin@nat@width>\linewidth\linewidth\else\Gin@nat@width\fi}
\def\maxheight{\ifdim\Gin@nat@height>\textheight\textheight\else\Gin@nat@height\fi}
\makeatother
% Scale images if necessary, so that they will not overflow the page
% margins by default, and it is still possible to overwrite the defaults
% using explicit options in \includegraphics[width, height, ...]{}
\setkeys{Gin}{width=\maxwidth,height=\maxheight,keepaspectratio}
% Set default figure placement to htbp
\makeatletter
\def\fps@figure{htbp}
\makeatother
\setlength{\emergencystretch}{3em} % prevent overfull lines
\providecommand{\tightlist}{%
  \setlength{\itemsep}{0pt}\setlength{\parskip}{0pt}}
\setcounter{secnumdepth}{-\maxdimen} % remove section numbering

\title{CTT Locator User Guide - New}
\author{David La Puma}
\date{3/12/2021}

\begin{document}
\maketitle

{
\setcounter{tocdepth}{2}
\tableofcontents
}
\hypertarget{the-ctt-locator}{%
\subsection{The CTT Locator}\label{the-ctt-locator}}

The CTT Locator is a portal device that can receive a variety of
wildlife tag signals. It is compatible with the entire radio tag line,
including ES200 devices.

\hypertarget{getting-started}{%
\section{Getting Started}\label{getting-started}}

\hypertarget{in-the-box}{%
\subsection{In the Box}\label{in-the-box}}

\begin{itemize}
\tightlist
\item
  Yagi Antenna
\item
  BNC to SMA coaxial adapter
\item
  SMA coaxial cable
\item
  CTT Locator with integrated Lithium Polymer battery and USB charger
  port
\end{itemize}

\hypertarget{powering-on-the-locator}{%
\section{Powering on the Locator}\label{powering-on-the-locator}}

To power on the Locator, press the switch to on. A blue light will
immedately shine, indicating the Locator has power. After a short bootup
sequence, the Locator will create a hotspot for your phone to connect
to. Due to technical limitations, Android phones will need to turn off
Cellular Data. This is not required on iPhone devices.

In addition to phones, the Locator can be used with laptops, tablets,
and other modern devices.

When the blue light flashes, the Locator is in its final stages of
booting. It will be ready in about 10 seconds after this indication.

Connect to a WiFi network created by the Locator.

Next, enter this full url into the browser. Note that the \url{https://}
is important:

\url{https://locator.click/}

In a few moments, a webpage will load and show the main locator screen.

\hypertarget{screen-overview}{%
\section{Screen Overview}\label{screen-overview}}

When you load the Locator screen, an interface will appear. You can
change the various modes of the locator by clicking on the icons at the
bottom.

First, we will go over the title bar:

The circle on the right will flash red when a tag is detected. A
rotating circle animation indicates that the locator is actively
connected to the phone. The far right icon indicates battery life, and
contains the voltage. Approximately 4.10 volts is a full battery, 3.7
volts is half full, and 3.4 volts is empty. This will also be indicated
by the battery slowly reducing, similar to your smart phone.

Next we will go over the icons at the bottom:

The ``Home'' icon shows all detected tags in a list, which is the
default view when you first connect to the device. For more details, see
the Home View section.

The ``Identify'' icon helps you identify unknown tags or verify a tag's
ID when it is in close range. For more details, see the Identify View
section.

The ``Favorites'' icon is a place where you can see the tags you have
specified as favorites. This is useful when a lot of other tags are
nearby and you wish to focus on a specific tag. For more details, see
the Favorites View section.

The ``Search'' icon is where you can search for tags within all of the
tags recently detected during your session. For more details, see the
Search View section.

The ``Download'' icon allows you to download all collected tag data as
well as ES200 data. For more details, see the Download View section.

The ``Settings'' icon allows you to control various locator settings and
options. For more details, see the Settings View section.

\hypertarget{home-view}{%
\section{Home View}\label{home-view}}

\includegraphics{\#home-view}
(\url{https://user-images.githubusercontent.com/1101026/105431612-ee816400-5c23-11eb-8a11-b1b6b08be0dd.png})

The home view shows all nearby tags. If a tag is first seen, it will
show up as red, to indicate a new tag has been detected. If the tag is
seen again, it will turn black. Each time the tag is detected, the text
will briefly flash yellow to indicate it has been updated.

\begin{figure}
\centering
\includegraphics{/Users/davidlapuma/Dropbox/CTT_Git/ctt_documentation/images/110704154-94803200-81c2-11eb-9a92-9746662533c3.png}
\caption{mainview-locator}
\end{figure}

\hypertarget{tag-view}{%
\section{Tag View}\label{tag-view}}

The tag view provides a signal strength indicator to help you localize a
tag. As the tag signal gets stronger, more bars appear. the negative
number under the signal bars is the signal strength in dBm, or decibel
milliwatts. The lower the number, the weaker the signal.

A large circle is at the top of the Tag View. This flashes when the tag
is received. Under this is the last time the tag was seen.

Touching or clicking the X in the top right corner closes tha tag view.

You can add a tag to your favorites list by tapping the ``Favorite''
button. Tapping this again removes the tag from your favorites.

\hypertarget{identify-view}{%
\section{Identify View}\label{identify-view}}

In the identify view, tags will appear if they have signals stronger
than -30 dBm. This would be a tag in very close proximity to the
locator.

\hypertarget{favorite-view}{%
\section{Favorite View}\label{favorite-view}}

Any tags that are selected as favorites will enter the favorite view.
You can click on any tag here to see the Tag View of any tag in this
list, just as you would in the main view.

\end{document}
